\chapter{INTRODUCTION}

Audio codecs are one of the main modules of many of digital systems dealing with audio. These algorithms have been designed to transform the audio signal to a bit stream and do the reverse transformation, from bits to audio. The main objective of the algorithm is fidelity to the original signal. Keeping the bit rate (length of the bit stream) and computational expensive of the transforms as small as possible are also desirable.\\
Since codec design is an old problem, there has been proposed a large set of techniques to address this problem \cite{acelp}, \cite{trans}. Despite the importance and complexity of the problem, there has been a narrow set of techniques which has been used repeatedly in different systems. Whereas the objective of an audio codec is to keep the similarity of input and output \textit{for human perception}; in most of commercial audio codecs, there are several blocks and solutions to enhance the perceptual quality of the output. Main tools for quality enhancement consists of bit budgeting and perceptual masking. These methods are based on empirical results in psychoacoustics.\\
The objective of this work is to address these issues. The first objective of this work is to, incorporate new methods and tools from machine learning. Machine learning was the center of attention in signal processing community and beats the results of other methods for recognition related tasks. Deep neural networks were the super star in the different application stages. One of the interpretations of the deep learning is \textit{representation learning}, i.e. instead of using an off-the-shelf feature extractors, the model learns the (sub-) optimal algorithm to extract proper feature for the specific task.\\
Autoencoders share some structural similarities with our design problems. A known  neural networks training algorithm such as a version of backpropagation can optimize the modified autoencoder parameters and represent the codec.\\
The second objective of the current work, is to develop and utilize a model for human perception of audio quality. Different model for human auditory system have been proposed. It is presumed that by the aim of new findings in neuroscience and functions of auditory cortex, better approximation of human perception of audio may lead to design better codec. There are different possible paths in design of functional human brain inspired assessment system.
There are simplified models of each step of auditory system from the external ear to the high auditory cortex. A smooth, i.e. continuous and differentiable model (function, stochastic process, etc.) can serve as the basis to calculate the error signal and update the parameters regards to that error signal.\\
Next chapter of the current document provides a review over techniques and methods that so far have been used in commercial codecs. Next a model of human auditory system will be addressed. In the last chapter the autoencoder based codec and the primary results of the model will be reported.