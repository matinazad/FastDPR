\chapter{CONCLUSION}

\section{Introduction}
There are different audio coding techniques that each of them performs its best for a
limited set of input signal (e.g. speech or audio) and a certain bit rate range.
Providing a comprehensive survey of audio codecs is not in the horizon of this text.

\section{Classification of Audio Codecs}
There are different classifications of audio codec. Forming a new paradigm for audio coding
is the promise of this document, the classification by coding techniques is provided here.
This classification is based on (

\subsection{Parametric Coders}
This codec is designed to provide intelligibility in very low bit rates. Parametric codec assumes a model for production of audio signal. Encoder extract the parameters of model
and decoder produce the signal by substituting the received parameters in the same model.
According to the assumptions about signal production, this codec performs well only for
a specific class of signals, e.g. speech signals.

\subsection{Waveform Coders}
These algorithms try to minimize the distance between input and output signals, so higher
bit rate is required.

\section{Hybrid coders}
These codecs advantage from both parametric codecs by taking a speech production model
into account, and enhance the quality by using richer models for excitations. Codebook

excitation is one of these methods which has been used in CELP codec.
The majority of modern codecs belong to hybrid compression systems.

\section{Main Techniques in Audio Codecs}

\subsection{Linear Prediction}
Linear Prediction Coding (LPC), was first proposed in \cite{} in 1971 for speech coding.
LPC is linear mathematical model of human vocal tract. \cite{} Generally it is formulated by
an all-pole filter. LPC filter extract spectral envelope of the speech. It assumes the input
signal (audio or speech in this case) are generated according to an autoregressive model.
Based on framing effect (setting samples values outside of the frame used for correlation
calculation to zero) this method add some inaccuracy to estimation even for autoregressive
model of the same order.
LPC finds the mean square error answer to the following equation.

\begin{equation}
P=
\begin{bmatrix} p_{11} & p_{12} & \cdots & p_{1r} \\
p_{21} & p_{22} & \cdots & p_{2r} \\
\vdots & \vdots & \ddots & \vdots \\
p_{r1} & p_{r2} & \cdots & p_{rr}
\end{bmatrix}
\end{equation}

$p + 1$ in formulas above indicates the order of the filter.
This problem can be solved for a using Levinson-Durbin algorithm in quadratic
 time.
 
\section{Perceptual Masking}
It has been known for a long time that mean square error (MSE) does not resemble perceptual
similarities. The nonequivalence roots on a phenomena called "masking". Masking,
models situations in which one sound overshadowed the other in human ear. It happens
based on limited temporal and spectral resolution of auditory system. Masking happens
when two similar signals (in spectral and/or temporal domain) present and the weaker
one turns inaudible. Based on similarity of signals in temporal or spectral domain, the
masking may be called temporal masking or spectral masking respectively.
There is a large number of studies in modeling and simulating and implementation of masking
and almost all codecs benefits from these results by dedicating less bits to inaudible
component of signal. This is known as perceptual bit-budgeting. \cite{} In \cite{}, perceptual
distance between two signals has been defined as Euclidean distance between the linear
mappings of the signals. This process also can be seen as applying the kernel method
where the kernel is 3-dimensional time-frequency curve described in \cite{}, an estimation of
this filter is shown in 2.1. This scheme of perceptual masking is very useful in perceptual
coding and transform coding.

\section{Human Auditory Model}
More complex mathematical encoding-decoding scheme let us have higher quality in lower
bit rate. The other important aspect of a "good" codec depends on how does it model the
human auditory system and incorporate this modeling in design of a better codec.
Studying and modeling of auditory system is quit old topic and almost all systems benefit
from the results of these studies to some extends 2.4.5. The physiologic and biological
analysis of human auditory system is outside of the scope of this work. For the sake of
pragmatic reasons, only contemporary mathematical models which are applicable in the
codec design process, modeling different stages of the auditory will be presented.
These steps have been studied and their mathematical model were included in various
existing audio systems.
The early stages of the model consists of cochlear filtering, hair cell transduction, the
phenomenon of lateral inhibition in the auditory neural pathway and midbrain integration. Mathematical models for auditory cortex have been relatively developed recently \cite{}.
Cochlea is modeled as a filter bank of constant-Q bandpass filters hcoch(t; f). As figure
2.3 indicates, the output of the cochlea filter bank will passed trough a model of auditory
nerve pattern. Inner hair cell itself is modeled by a three-step process, first, high-pass filter
tycoch(t; f) then the nonlinearity ghc(:) (compression) and at the end a low-pass leakage
filter hc(t). Cochlear-nucleus neurons are modeled as derivative with respect to frequency
fyAN(t; f) followed by a rectifier max(tycoch(t; f); 0). The last step of the first section of
the model is a short-term integrator midbrain(t; f ) with time constant  , that models the
further loss of phase-locking observed in the midbrain. \cite{}
The second part of the model which is also one of the focus point of this project is the
model of auditory cortex have been developed and improved vastly in recent years by
conducting experiments on humans and more on animals \cite{}. Basically the current model
of assumes that audio (output of the early auditory stage) based on specific properties
makes special regions of the auditory cortex fire the most. These spatial varying firing rate
leads to Spectro Temporal Response Filters or STRFs. STRFs have been modeled
as a redundant two to three or four dimensional mappings (based on the notation and
not much difference on implication). The first models are three dimensional wavelet like
filters which analyses the spectrogram of the input (audio) signal to three components
called rate (temporal modulation), scale (frequency modulation) and direction (the signal
is upward in tone or downward). Further studies try to make this model more accurate by
adding extra boundaries our condition to it \cite{}.
